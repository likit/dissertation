\noindent
\textbf{Alternative Splicing}: A mechanism that produces a
variety of mRNA variants from a single gene by removing introns
and joining exons in different manners. Alternative splicing is
tightly controlled and can be stimulated by either internal or
external stimuli.

\noindent
\textbf{{\em de novo} Assembly}: A method used to generate a
longer transcript or a contig by concatenating overlapping short
reads without the use of a reference genome or transcriptome as a
guide.

\noindent
\textbf{Differential Gene Expression}: A statistically significant
difference in expression of a gene between samples.

\noindent
\textbf{Differential Exon Usage}: A statistically significant
difference in expression of an exon between samples.

\noindent
\textbf{Global Assembly}: See {\em de novo} Assembly.

\noindent
\textbf{Indel}: Insertion or deletion of nucleotides or amino
acids.

\noindent
\textbf{K-mer}: A subsequence (of length k) from a read obtained
from DNA or mRNA sequencing, where k is an integer.  An example
of k=4 or 4-mers is ACGT. 

\noindent
\textbf{KEGG Pathway Annotation}: A collection of manually
curated pathway maps representing connections and interactions of
genes and molecules involved in biological pathways, diseases,
drugs, and chemical substances.

\noindent
\textbf{Local Assembly}: A technique used to increase sensitivity of
splice variant detection and facilitate assembly. In contrast to
global assembly, only reads mapped to a genome are assembled.

\noindent
\textbf{Paired-End Reads}: A pair of short reads with a specific
insert size (a distance between mates) generated by next-generation
sequencing technology. Two mates are generally sequenced from
opposite directions.

\noindent
\textbf{Pipeline (computing)}: A set of commands and tools used
for computational analyses that are usually organized in a
specific order or series, where the output of one tool/command is the
input of next one.

\noindent
\textbf{Polymorphism}: A variation of nucleotides or amino acids
between two or more DNA, mRNA and protein sequences.

\noindent
\textbf{Read Mapping}: Alignment of reads to a genome or a reference
set of transcripts.

\noindent
\textbf{Short Reads}: A short stretch of nucleotides (50-400bp)
produced by next-generation sequencing technology.

\noindent
\textbf{Single-End Reads}: A short reads produced by next-generation
sequencing technology, usually sequenced from one strand of a DNA
or an mRNA.

\noindent
\textbf{Transcriptome}: Total mRNA molecules expressed from
tissues or cells.
