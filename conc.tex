\chapter{Conclusions}
\section{Summary}

The goal of the projects discussed in this dissertation is
to identify genes and isoforms that contribute to MD
resistance using RNA-Seq data. However, available chicken
gene models are not suitable for expression analysis because
they are not complete. For example, some of them do not
include full UTRs and some exons expressed. We have
demonstrated that different gene models yield different results
in differential expression analysis as well as biological
pathway analysis. Therefore, it is important to improve on
existing gene models using RNA-Seq data.

We have developed a pipeline that can combine both {\em de
novo} assembly and reference-based assembly to construct
gene models that we have shown to recover more splice
variants in the dataset than either method alone. We have
also described the local assembly technique that is more
memory efficient than global assembly (conventional {\em
de novo} assembly) and can recover some unique splice
variants not found by conventional assembly.

We used the merged gene models from Cufflinks and {\em de
novo} assembly described above to study gene and isoform expression
from RNA-Seq data collected from spleens of resistant and
susceptible inbred chicken lines (lines 6 and 7
respectively) at 4 dpi -- lytic phase of infection. We found
that more genes were differentially expressed in the
susceptible line than the resistant line and those DE genes were
enriched in genes active in both the innate and the adaptive immune system.
In contrast, only pathways involved in the innate immune
system were enriched in DE genes within the resistant line.
The results indicate that the adaptive immune responses were enhanced in the
susceptible line at this stage of infection. The adaptive
immune responses involve recruiting and activating T cells,
which results in more target T cells for MDV because only
activated T cells are thought to be infected by MDV.  More
infected T cells could lead to a larger number of T cells
that transform into T cells lymphoma in the susceptible line.

To investigate further, we used our merged gene models to
study isoform expression between susceptible and resistant
lines. Splice variants are highly important in the immune
system and many genes have been shown to play a role in
increasing risk and prognosis of diseases. Several mutations
at binding motifs of splicing factors that alter splicing
patterns and are associated with diseases have also been
characterized. However, reads from RNA-Seq are too short to
span across the entire transcripts; therefore, it is
difficult to estimate isoform expression accurately.
Moreover, we cannot be certain that our gene models include
all expressed isoforms due to the method used to build gene
models from short reads. For these reasons, we decided to
use an exon centric approach, which only compares exon
expression. Exon expression can be estimated more accurately
because reads can span across exon junctions of an exon of
interest and its adjacent exons.

Using this method, we identified many genes that have
differential exon expression between susceptible and
resistant lines. Intriguingly, we found that some genes
have different patterns of exon expression between lines and
they were involved in the actin cytoskeleton pathway and some
other pathways involved in immune responses.
The actin cytoskeleton pathway is important in cell-to-cell
contact, cytokinesis, phagocytosis, and antigen presentation
which are all important in adaptive immune responses.
Based on the results, we hypothesized that splice
variants may play a role in limiting or controlling the
elicitation of the adaptive immune response in the resistant
line, which in turn limiting the spread of the virus to
activated T cells. Results from splicing finder prediction
also showed that exonic SNPs between susceptible and
resistant lines may regulate splicing patterns by disrupting
or creating new binding sites for splicing factors. Results
from our study, after further validation, could be used to
facilitate the marker-assisted selective breeding to develop
chickens with genetic resistant to MD.

\section{Future work}

Results from our study have shown that isoforms could play a
significant roles in MD resistance. Therefore, to better
understand a mechanism that controls MD resistance,
especially at an isoform-level, we need to obtain more
sequencing data to identify more isoforms involved in immune
responses to MDV infection. Ideally, we need to obtain data
from multiple time points and with longer read lengths.

However, obtaining more data is relatively inexpensive in
regard to time and labor spent on analyzing data.  To
enhance sensitivity of gene and isoform detection, gene
models need to be updated using new data to ensure that the
models include all genes and isoforms expressed in the
dataset. To our knowledge, there is no available integrated
system for building gene models from multiple methods and
existing gene models, and identifying alternative exon usage and
exonic SNPs, as well as predicting protein domains from exons
of interest as described in this dissertation. The whole
process is complicated and only practical for adept
bioinformaticians.  To accommodate biologists, we plan to
develop an open protocol for isoform analysis based on the
pipeline used in this dissertation from the beginning to the
end. Every step in the protocol will be thoroughly
documented and the protocol will be available online and can
be updated by users.  Once established, the protocol can be
used by biologists to iteratively update gene models and
perform gene and isoform expression analysis from RNA-Seq
data from chickens as well as other organisms with
low-quality gene annotation.

The local assembly described in this dissertation can
recover unique splice variants not found in the global
assembly. However, the method relies on read alignment to a
reference genome. Therefore, it is dependent on the quality
of the genome as well as efficiency of the short read aligner
used. In addition, for most non-model organisms, a
high-quality reference genome is not available; thus, this
method cannot be employed. Even though we have successfully
used the method to recover many splice variants, the
inherent mechanism of local assembly has not been
understood.  An understanding of the mechanism could
lead to an algorithm that improves sensitivity of splice
variants detection from {\em de novo} assembly and can be
used in organisms without a reference genome.
