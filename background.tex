\chapter{Introduction}
\section{Overview}

Marek's disease (MD) is a lymphoproliferative disease of chickens
caused by a highly oncogenic Marek's disease virus (MDV).
Vaccination has been effective in reducing tumor formation and
loss from MD; however, it does not confer protective immunity
against infection and shedding of MDV~\cite{gimeno2008marek}.
Continuous shedding of viruses from feather follicles of
vaccinated chickens allows field viruses to circulate in the
flock~\cite{morrow2004marek}.  As a consequence, more virulent
strains that overcome vaccinal protection have
emerged~\cite{witter1998control,gimeno2008marek} leading to
outbreaks that have cost an estimate of \$2
billion~\cite{morrow2004marek}.  Selective breeding of resistant
birds is an alternative control measure against MD.  In this
study, we use transcriptome data from next-generation sequencing
to compare gene and isoform expression in response to MDV
infection between MD resistant (line 6) and susceptible (line 7)
chickens in order to identify candidate genes and isoforms that
contribute to resistance to MD.

\section{Background}
\subsection{Genetic resistance to MD}

Genetic resistance to MD can be categorized into {\em MHC}
and {\em non-MHC} associated forms.  The {\em MHC} locus (B
system haplotypes) is strongly associated with MD resistance
and the phenotypes of different haplotypes have been well
characterized.  For example, B\textsuperscript{21},
B\textsuperscript{2} and B\textsuperscript{6} alleles are
usually associated with MD resistance, on the other hand,
B\textsuperscript{5}, B\textsuperscript{13} and
B\textsuperscript{19} are associated with susceptibility.
Chickens with homozygous B\textsuperscript{21} were most
resistant to MD ($0\%$), whereas, chicken heterozygous with
other alleles and B\textsuperscript{21} developed $40-93\%$
MD~\cite{briles1980identification}.  However, in some
populations, the association of B haplotypes does not follow
the aforementioned patterns. For example, the Cornell
chicken strains C and K were selected for resistance to MD,
yet strain C possesses B\textsuperscript{6},
B\textsuperscript{13} and B\textsuperscript{15} and strain K
possesses
B\textsuperscript{15}~\cite{bacon1987influence,briles1980identification}.
Similarly, inbred chicken lines 6 and 7 selected for
resistance and susceptible to MD are both
B\textsuperscript{2,2} homozygous.  This suggests that genes
outside the {\em MHC} locus also contribute to MD
resistance.

It has been postulated that different mechanisms are
controlled by {\em MHC} and {\em non-MHC}
loci~\cite{kaiser2003differential}.  For instance, Kaiser et
al.~\cite{kaiser2003differential} reported that the level of
MDV viral load also differed between lines 6 and N after 10
and 14 days post infection (dpi) -- the onset of latency --
and decreased to the same level by 21 dpi.  Although both
lines 6 and N are resistant to MD, the level of MDV viral
load in line 6 was significantly higher than in line N at 10
and 14 dpi (P$<0.05$).  Resistance line N is associated
with the B\textsuperscript{21} haplotype, whereas, the
resistance of line 6 is associated with the MDV1 locus.

% {\em non-MHC} genes are thought to have effects in cellular interaction,
% cytokines or innate immunity.
% {\em In vitro} and {\em in vivo} studies have shown that T cells from line 7
% absorbed a greater number of MDV and were more active to mitogen than those from
% line 6~\cite{lee1983ontogeny,powell1982mechanism}.
% It has been postulated that the resistance is attributable to an inherent
% mechanism that limits the susceptibility and the number of target T cells.

\subsection{Genome-wide scan for genes conferring resistance to MD}

Identification of precise genes is essential for developing
resistant breeds using a marker-assisted selection (MAS)
.  A genetic-driven method has been used to identify
quantitative trait loci (QTL) associated with MD resistance
or MD infection using lines 6 and 7 as parents.  Seven
significant and seven suggestive QTLs were identified by
Vallejo ~\cite{vallejo1998genetic} and
Yonash~\cite{yonash1999high}; whereas,
Bumstead~\cite{bumstead1998genomic} identified a significant
locus on chromosome 1, which is referred to as MDV1.  MDV1
region has a strong association with resistance in terms of
reduced viral level and tumour incidence.  Based on
comparative mapping analysis, this locus has shared
synteny with the mouse region that includes the {\em CMV1} and {\em
Ly49} genes.  {\em CMV1} controls resistance to murine
cytomegalovirus by affecting viral proliferaton and {\em
Ly49} serves as an inhibitor of cell killing by NK cells.
Correlation between functions of {\em CMV1} and {\em Ly49}
and resistance associated to this region is notable.

The drawback of QTL analysis of complex traits is it
requires a large population size ($>500$) to achieve
reliable results.  Ideally, a population size of $>1000$ is
needed for fine-mapping suitable for successful MAS ($<1$
cM)~\cite{young1999cautiously}.  Furthermore, only a few true
QTL of major effect can be detected in any given study
~\cite{collard2005introduction}.  Therefore, an alternative
approach that could facilitate the identification of precise
genes at a genome-wide scale is warranted.

Differential gene expression is thought to contribute to
phenotypic variation and difference in gene expression can
be controlled by genetic factors~\cite{morley2004genetic}.
Furthermore, functional analysis of differential-expressed
(DE) genes can provide an insight into pathways and
biological mechanisms that control phenotypes.  Data from
RNA-Seq can also be used to investigate mutations such as
SNPs and indels that might be responsible for phenotypic
differences.

Recently, high-throughput technologies including microarray
and next-generation sequencing (NGS) have been used to
compare gene or EST expression between MDV-infected birds of
selected resistant and susceptible lines to identify
candidate genes that contribute to MD resistance.
% need to add more about this?
Sarson~\cite{sarson2008transcriptional} used a low-density
immune-specific cDNA microarray to compare expression of
immune genes from splenocytes between resistant and
susceptible chickens, with B\textsuperscript{21} and the
B\textsuperscript{19} haplotype respectively, and found that
the percentage of differential-expressed genes between
B\textsuperscript{19} control and infected birds was
highest.  Furthermore, B cell surface markers such as {\em
Bu-1} and {\em IgM} levels were downregulated, which might
contribute to the decrease in B cells in the lytic phase of
infection.   Morgan et al~\cite{morgan2001induction} also used a DNA
microarray designed for chicken to compare the differential
expression between lines 6 and 7 at multiple time points.
Of $\sim1200$ genes and ESTs, a few genes including growth
hormone ({\em GH1}) and lymphotactin ({\em SCYC1}) were
found to be differentially expressed ($>2$ fold) and located
near the QTL region on chromosome 1 (MDV1) identified by
Bumstead~\cite{bumstead1998genomic}.  Growth hormone binds
to MDV {\em SORF2} and is associated with MD
resistance~\cite{liu2001growth} and lymphotactin serves as a
chemoattractant of CD4+ and CD8+ T cells.

The findings from these studies are interesting, yet very
limited due to the small scale of the analysis.  A
genome-wide scale microarray study was, therefore, conducted
by Smith et al~\cite{smith2011systems} to compare gene expression
between virus-infected lines 6 and 7 from spleen and thymus
at a larger scale.  In control groups, 395 genes in spleen
and 177 genes in thymus were differentially expressed
between lines 6 and 7.  Genes highly expressed in line 6 and
involved in innate immune responses include {\em DNAJC3,
DDT, NMU, GSTO1, VIP, HPS5, MMP7, FGFR3, HSCB, E2F4, SFTPA2}
and {\em GNG12}.  In infected groups, 593 genes in spleen
and 156 genes in thymus were differentially expressed
including {\em IgG-H, AMIGO2, MMP13}, and {\em CLEC3B} that
were highly expressed in line 6 and {\em AVD, IRG1, HSP25,
ART1, IL-18, NOS2A, CXCL13, CCLi2, MX1, SOCS1}, and {\em
IL-6} that were highly expressed in line 7.  Approximately
$26-30\%$ of DE genes were located in the previously reported QTL
regions, which could be potential candidate genes.

To conclude, high-throughput technologies together with
genetic approaches have been successfully used to identify
many candidate genes associated with MD resistance, which
confirms the complex nature of MD genetic resistance.
However, the interaction of these candidate genes and their
roles in disease resistance have not been fully defined.
Furthermore, only global gene expression has been used for
identification of genes conferring MD resistance, whereas,
alternative isoforms have been shown to play a significant
role in immune responses and contribute to disease
susceptibility~\cite{lynch2004consequences,wang2007splicing}.
This dissertation aims to identify both candidate genes and
isoforms that are involved in resistance to MD.

\section{Problem Statement}

In the immune system, many genes can express different
isoforms with a distinctive, synergistic or even opposing
function~\cite{lynch2004consequences,wang2007splicing}.
Expression of isoforms is regulated in part by {\em
cis-}regulatory sequences within an exon or intron of a
pre-mRNA~\cite{blencowe2000exonic}.  Although expressions of
various genes have been examined across stages of infection
and various genetic
background~\cite{smith2011systems,sarson2008transcriptional,kaiser2003differential,morgan2001induction},
the investigation of genome-wide isoform expression has not
been conducted.  The gap in the knowledge is, therefore, the
identification of candidate isoforms that contribute to MD
resistance.

Many tools are currently available for estimating gene and
isoform expression from RNA-Seq data as well as comparing
their expressions between samples (reviewed in
~\cite{trapnell2012differential}).  However, they are all
reliant on completeness of provided gene models.  Available
gene models such as Ensembl annotation do not include all
genes and isoforms; therefore, they are not ideal for
identification of candidate genes and isoforms.  In
addition, some annotated gene models do not included
unstranslated regions (UTRs) that can have important
biological functions as well as significant sizes.  Methods
for expression estimation typically infer gene expression
from the number of reads mapped to transcripts or gene
models.  Without complete gene models, the expression will
be inaccurately estimated resulting in errors in
differential expression (DE) prediction.  Consequently,
biological pathway prediction will be affected because the
prediction is based solely on results from DE prediction.

\section{Significance of Research}

The study will create a pipeline to integrate RNA-Seq data
with existing gene models to extend the models so that they
include genes, isoforms, and UTRs expressed in a sample.
The extended genes models will allow a better estimate of
genes and isoforms expression, which will be used to predict
biological pathways perturbed in responses to MDV infection.
Comparison of perturbed biological pathways between
resistant and susceptible birds will provide an insight into
genes and mechanisms that contribute to MD resistance.
Additionally, in depth investigation of isoform expression
could lead to identification of exonic SNPs that regulate
alternative splicing patterns, which could reveal
unprecedented level of molecular and genetic mechanisms that
impart resistance to MD.

\section{Outline of Dissertation}

In the first chapter, we describe a method developed to
construct gene models from different sources including {\em
de novo} assembly of short reads from RNA-Seq data, a
reference-guided assembly (Cufflinks) and Ensembl gene
models. We show that our method can be used to combine gene
models from different sources to build gene models that
include more splice variants. We also describe a local
assembly method that can enhance sensitivity of splice
variant detection. In the second chapter, we compare
results of gene expression and KEGG pathway analysis from
different gene models. We demonstrate that different gene
models give different results from pathway enrichment
analysis. In addition, we discuss the use of combined
annotation from chicken and mouse to increase sensitivity of
pathway prediction. In the last chapter, we report
differentially expressed genes and isoforms and discuss a
potential role of differentially expressed isoforms in MD
resistance.
